% !TeX encoding = UTF-8
\documentclass[13pt]{article}
\usepackage[utf8]{inputenc}
\usepackage[english,russian]{babel}
\usepackage[T2A]{fontenc}
\usepackage{tempora}
\usepackage{indentfirst}

\usepackage{amsmath,amsfonts,amssymb,amsthm,mathtools} % AMS
\usepackage{icomma}
\usepackage{mathrsfs}

\usepackage{graphicx}
\usepackage{fancyhdr}
\usepackage{listings}
\usepackage{xcolor} % for setting colors
\lstset{
	basicstyle=\scriptsize\ttfamily,
	commentstyle=\ttfamily\color{gray},
	numbers=left,
	numberstyle=\ttfamily\color{gray}\footnotesize,
	stepnumber=1,
	numbersep=5pt,
	backgroundcolor=\color{gray!10},
	rulecolor=\color{black!30},
	showspaces=false,
	showstringspaces=false,
	showtabs=false,
	frame=single,
	tabsize=2,
	captionpos=b,
	breaklines=true,
	breakatwhitespace=false,
	title=\lstname,
	escapeinside={},
	keywordstyle={},
	morekeywords={},
    extendedchars=false,
    inputencoding=utf8
}

\usepackage[left=3cm,right=1.5cm,top=2cm,bottom=2cm]{geometry}
\author{Игорь Комолых, Сергей Лущик}
\title{Лабораторные работы по курсу "Параллельное~и~распределённое~программирование"}
\date{\today}

\begin{document}
	\begin{titlepage}
		\maketitle
	\end{titlepage}
	
	\section{Умножение матриц}\label{sec:lab1}

	Эта лабораторная работа заключалась в сравнении последовательной и параллельной реализации алгоритмов умножения матриц,
	а так же в сравнении времени работы программы при разных способах обхода массива.

	В результате выполнения работы были получены:
	\begin{itemize}
		\item описанный класс Matrix
		\item bash и sbatch файлы запуска программы на персональных компьютерах и кластере САФУ
		\item python-скрипт для построения графиков на основе полученных данных.
 	\end{itemize}

	Результатом выполнения программы является строка, в которой через запятую указаны количество используемых потоков, размерность квадратной матрицы, время работы. При запуске bash или sbatch сценариев, происходит формирование CSV-файла, по данным которого в дальнейшем можно строить графики ускорения, эффективности и времени работы (см. Рис.~\ref{fig:beforeChange} и~\ref{fig:afterChange}). Программы собирались и запускались на вычислительном кластере САФУ.

	\begin{figure}[h]
		\centering
		\includegraphics[scale=.5]{result}
		\caption{графики до смены порядка обхода матриц.}
		\label{fig:beforeChange} 
	\end{figure}

	\begin{figure}[h]
		\centering
		\includegraphics[scale=.5]{result_trs}
		\caption{графики после смены порядка обхода матриц.}
		\label{fig:afterChange} 
	\end{figure}

	По рис.~\ref{fig:beforeChange} и~\ref{fig:afterChange} видно, что после изменения порядка обхода матрицы с привычного ``строки-столбцы'' на ``столбцы-строки'' (см. Листинг~\ref{lst:matrixLookup}) время работы программы может сокращаться в 2 и более раза, в некоторых случаях удавалось достичь ускорения в 4-5 раз.

	Данный пример демонстрирует особенности устройства кэша процессора и оперативной памяти. При обращении к какой-либо ячейке памяти, в кэш вместе с ней загружаются и несколько соседних ячеек. При обращении в порядке ``строки-столбцы'' два элемента, над которыми производятся операции в смежных итерациях алгоритма, в памяти будут находиться на расстоянии, равном размеру строки матрицы. Если же обходить массивы в порядке ``столбцы-строки'', смежные итерации будут оперировать элементами одной строки матрицы, элементы которой располагаются в памяти друг за другом.
	
	\begin{lstlisting}[language=C++, caption={Два способа обхода матрицы}, label={lst:matrixLookup}, texcl=true]
	//
	// строки-столбцы
	//
				#pragma omp parallel for shared(result, first, second)
				for (size_t i = 0; i < result.rows(); ++i) 
					for (size_t j = 0; j < result.cols(); ++j) {
						result(i, j) = 0;
						for (size_t k = 0; k < result.rows(); ++k) 
							result(i, j) += first(i, k) * second(k, j);
					}
					
		//
		// столбцы-строки
		//

				#pragma omp parallel for shared(result, first, second)
				for (size_t j = 0; j < result.cols(); ++j) 
					for (size_t i = 0; i < result.rows(); ++i) {
						result(i, j) = 0;
						for (size_t k = 0; k < result.rows(); ++k) 
							result(i, j) += first(i, k) * second(j, k);
					}

	\end{lstlisting}
	\section{Задача Дирихле для уравнения Пуассона}\label{sec:lab2}

	С использованием класса матриц, полученного в ходе выполнения первой лабораторной были реализованы последовательный и параллельный алгоритмы решения задачи Дирихле для уравнения Пуассона, написаны python-скрипты для построения графиков поверхностей по полученным данным.
	Результат тестового запуска алгоритма для уравнения $f(x,y) = 4$ с краевыми условиями $g(x,y) =(x - 0.5)^2 + (x - 0.5)^2$ представлен на рис.~\ref{fig:surfPlot}.
	\begin{figure}[h]
		\centering		\includegraphics[scale=0.5]{surface_plot}
		\caption{График решения задачи Дирихле для уравнения Пуассона.}
		\label{fig:surfPlot} 
	\end{figure}

	Для того чтобы гарантировать получение точно таких же решений, как и в непараллельном алгоритме Гаусса-Зейделя, параллельный алгоритм был построен по волновой схеме. Для вычисления начения текущего элемента $U_{i,\,j}$ алгоритм Гаусса-Зейделя использует два ранее вычеслинных элемента $U_{i-1,\,j}$, Вначале, таким условиям удовлетворяет только элемент $U_{1,\,1}$, однако, после его подсчета становится доступна следующая диагональ $U_{2,\,1} - U_{1,\,2}$. Получаем,	что выполнение одной итерации можно разбить на последовательность шагов, на каждом из которых вычисляются узлы, расположенные на одной из диагоналей исходной сетки. 
	
	\begin{lstlisting}[language=C++, caption={Два способа обхода матрицы}, label={lst:waveDirichlet}, texcl=true]
DirichletResult solveDirichlet(size_t N,  double eps) {
	auto startTime = std::chrono::steady_clock::now();
	
	Matrix u_mat(N+2,  N+2);
	Matrix f_mat(N,  N);
	double h = 1.0 / (N + 1);
	
	// Заполнение матриц  u\_mat, f\_mat начальными значениями 

	double max, u0, d;
	size_t i = 0, j = 0, iterations = 0;
	std::vector<double> mx(N+1);
	do {
		iterations++;
		// нарастание волны (k - длина фронта волны)
		for (size_t k = 1; k < N+1; k++) {
			mx[k] = 0;
			#pragma omp parallel for shared(u_mat, k, mx) private(i, j, u0, d) schedule(static, 1)
			for (i = 1; i < k+1; i++) {
				j = k + 1 - i;
				u0 = u_mat(i, j);
				u_mat(i, j) = 0.25 * (u_mat(i-1, j) + u_mat(i+1, j) + u_mat(i, j-1) + u_mat(i, j+1) - h*h*f_mat(i-1, j-1));
				d = std::fabs(u_mat(i, j) - u0);
				if (d > mx[i]) mx[i] = d;
			}
		}
		for (size_t k = N-1; k > 0; k--) {
			#pragma omp parallel for shared(u_mat, k, mx) private(i, j, u0, d) schedule(static, 1)
			for (i = N-k+1; i < N+1; i++){
				j = 2*N - k - i + 1;
				u0 = u_mat(i, j);
				u_mat(i, j) = 0.25 * (u_mat(i-1, j) + u_mat(i+1, j) + u_mat(i, j-1) + u_mat(i, j+1) - h*h*f_mat(i-1, j-1));
				d = std::fabs(u_mat(i, j) - u0);
				if (d > mx[i]) mx[i] = d;				
			}
		}
		max = 0;
		for (i = 1; i < N+1; i++) {
			if (mx[i] > max) max = mx[i];
		}
	} while (max > eps);

	auto runtime = std::chrono::steady_clock::now();
	auto runtimeDuration = std::chrono::duration_cast<std::chrono::duration<double>>(runtime - startTime);
	DirichletResult result(omp_get_max_threads(), u_mat, iterations, runtimeDuration.count(), eps);
	return result;
}
	\end{lstlisting}
	
	Реализация алогитма, представенного на листинге \ref{lst:waveDirichlet}, была скомпилирована и запущена на вычислительном кластере САФУ. При замере времени работы программы на различном числе ядер, были получены графики, изображенные на рисунке \ref{fig:dplots}.
	
	\begin{figure}[h]
		\centering		\includegraphics[scale=0.5]{d_plots}
		\caption{графики ускорения, эффективности и времени работы волнового алгоритма.}
		\label{fig:dplots} 
	\end{figure}
	
	Сравнение результатов этой и предыдущей лабораторной работы наглядно демонстрирует, что все алгоритмы обладают разной эффективностью и по-разному переносят распараллеливание. 

\end{document}